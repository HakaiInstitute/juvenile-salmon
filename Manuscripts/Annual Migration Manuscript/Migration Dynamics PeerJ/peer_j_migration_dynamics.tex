%% Submissions for peer-review must enable line-numbering
%% using the lineno option in the \documentclass command.
%%
%% Preprints and camera-ready submissions do not need
%% line numbers, and should have this option removed.
%%
%% Please note that the line numbering option requires
%% version 1.1 or newer of the wlpeerj.cls file, and
%% the corresponding author info requires v1.2

\documentclass[fleqn,10pt]{wlpeerj} % for preprint submissions

% ZNK -- Adding headers for pandoc

\setlength{\emergencystretch}{3em}
\providecommand{\tightlist}{
\setlength{\itemsep}{0pt}\setlength{\parskip}{0pt}}
\usepackage{lipsum}
\usepackage[unicode=true]{hyperref}
\usepackage{longtable}



\usepackage{lipsum} \usepackage{multirow} \usepackage{float}
\floatplacement{figure}{H}

\title{Juvenile Salmon Migration Dynamics in the Discovery Islands and
Johnstone Strait in 2018}

\author[1]{Brett T. Johnson}

\corrauthor[1]{Brett T. Johnson}{\href{mailto:brett.johnson@hakai.org}{\nolinkurl{brett.johnson@hakai.org}}}
\author[]{Julian C.L. Gan}

\author[]{Carly V. Janusson}

\author[1, 2, 3]{Brian P.V. Hunt}


\affil[1]{Hakai Institute Quadra Island Ecological Observatory, Heriot Bay, BC V0P
1H0}
\affil[2]{Institute for the Oceans and Fisheries, University of British Columbia
Vancouver, B.C., Canada V6T 1Z4}
\affil[3]{Department of Earth, Ocean and Atmospheric Sciences, University of
British Columbia Vancouver, B.C., Canada V6T 1Z4}


%
% \author[1]{First Author}
% \author[2]{Second Author}
% \affil[1]{Address of first author}
% \affil[2]{Address of second author}
% \corrauthor[1]{First Author}{f.author@email.com}

% 


\begin{abstract}
The majority of out-migrating juvenile Fraser River salmon
(\emph{Oncorhynchus} spp.) pass northwest through the Strait of Georgia,
the Discovery Islands, and Johnstone Strait. The Discovery Islands to
Johnstone Strait leg of the migration is a region of poor survival for
juvenile salmon relative to the Strait of Georgia. To better understand
the factors that are driving early marine survival through this region
the Hakai Institute Juvenile Salmon Program monitors key aspects of this
migration. Here we report on the 2018 migration in comparison to
averages from the 2015--2018 study period, which we use to define
`normal' in our building time series. In 2018 sockeye (\emph{O. nerka}),
pink (\emph{O. gorbuscha}), and chum (\emph{O. keta}) all migrated
earlier than normal. The median capture date in the Discovery Isalnds
was May 23rd for sockeye, five days earlier than normal; and June 12 for
pink and chum, which is five days earlier for pink and three days
earlier than normal for chum. Sea lice prevalence was lower than normal
for sockeye, pink, and chum. Notably, there were no \emph{Lepeophtheirus
salmonis} sea lice observed in Johnstone Strait in 2018. Sockeye were
longer than normal in 2018 whereas pink and chum were smaller than
normal. Sea surface temperatures in May and June were the warmest on
record in the study period (2015--2018). Pink salmon dominated the catch
in 2018, followed by chum, and then sockeye.
% Dummy abstract text. Dummy abstract text. Dummy abstract text. Dummy abstract text. Dummy abstract text. Dummy abstract text. Dummy abstract text. Dummy abstract text. Dummy abstract text. Dummy abstract text. Dummy abstract text.
\end{abstract}

\begin{document}

\flushbottom
\maketitle
\thispagestyle{empty}

\section*{Introduction}\label{introduction}
\addcontentsline{toc}{section}{Introduction}

The first months after marine entry have been identified as a
potentially critical period (R J Beamish and Mahnken 2001) for salmon
stock recruitment, which may ultimately be responsible for inter-annual
variability and long term declines in salmon stocks in British Columbia
(R. M. Peterman et al. 2010; R. J. Beamish et al. 2012). Pathogens,
parasites, predators and the impacts of climate change on food web
dynamics have emerged as leading causes for the decline. The Hakai
Institute Juvenile Salmon Program has been monitoring juvenile salmon
migrations in the Discovery Islands and Johnstone Strait (Figure
\ref{fig:map}) since 2015 in an effort to understand what factors may be
influencing early marine survival of sockeye, pink, and chum (Hunt et
al. 2018). This report summarizes migration timing, fish length,
parasite loads, species composition, and sea-surface temperature
observed from the first 4 years of this research and monitoring program.
These estimates will provide the context from which to investigate
questions and interpret results related to growth, survival, and the
conditions salmon experience during their migration through this
critical region.

\begin{figure}[H]

\includegraphics[width=0.8\linewidth]{map} \hfill{}

\caption{Sampling locations in 2018}\label{fig:map}
\end{figure}

\section*{Methods}\label{methods}
\addcontentsline{toc}{section}{Methods}

\subsection*{Field methods}\label{field-methods}
\addcontentsline{toc}{subsection}{Field methods}

See Hunt et al. (2018) for a detailed description of field and lab
methods. Briefly, we collect juvenile salmon weekly from the Discovery
Islands and Johnstone Strait during their northward migration from the
Strait of Georgia to Queen Charlotte Strait near northern Vancouver
Island, British Columbia. Sampling is conducted from May to July each
year since 2015 using purse seine nets (bunt: 27 m x 9 m with 13 mm
mesh; tow: 46 m x 9 m with 76 mm mesh). We sample in nearshore marine
habitats with depth \textgreater{} 10 m and effectively sample sockeye
(\emph{Oncorhynchus nerka}), pink (\emph{O. gorbuscha}), chum (\emph{O.
keta}) and incidentally capture coho (\emph{O. kisutch}), chinook
(\emph{O. tshawytschya}) and Pacific herring (\emph{Clupea pallasii}).
All animal care was in accordance with Animal Care Guidelines under
permit A16-0101. Temperature data were collected by deploying an RBR
conductivity, temperature, and depth profiler to depths \textgreater{}
30 m at station QU39 (Figure \ref{fig:map}) in the northern Strait of
Georgia.

\subsection*{Statistical methods}\label{statistical-methods}
\addcontentsline{toc}{subsection}{Statistical methods}

All metrics reported are in relation to the time series average
(2015-2018). The mean for each parameter of interest was calculated for
all years combined, and the z-score was calculated for each parameter to
determine the number of standard deviations away from the mean a given
parameter was in each year.

Annual migration timing for each species was measured by calculating the
median date of capture in the Discovery Islands, the date at which 50
percent of the fish passed through the region. To visualize migration
timing we plotted cumulative catch abundance between May 1st and July
9th each year and fit a logistic growth line. Species proportions were
calculated by dividing the total number of each species caught that
season across all seines by the sum of all species caught that season.
Fork length distributions were visualized by calculating kernel density
estimates from fork length data. The prevalence, intensity, and
abundance were of sealice was calculated as detailed in Margolis et al.
(1990). The mean sea surface temperature was calculated from the top 30
m of the water column in May and June from all years. To visualize
temperature anomalies we applied a loess regression to sea surface
temperatures from all four years to develop a model that would represent
the seasonal trend.

\section*{Results and Discussion}\label{results-and-discussion}
\addcontentsline{toc}{section}{Results and Discussion}

\subsection*{Migration Timing}\label{migration-timing}
\addcontentsline{toc}{subsection}{Migration Timing}

The peak migration date for sockeye in the Discovery Islands occurred 5
days earlier than average in 2018 (Z = -0.71). (Figure \ref{fig:mt}).
The median date of capture for 2018 sockeye was May 22nd, whereas the
time series average was May 27th. Conversely, the 2018 sockeye migration
occurred later than normal in Johnstone Strait, where the median date of
capture was June 6th, which is two days later than the time-series
average. See Table \ref{tab:mtdi} and Table \ref{tab:mtjs} for the
interqurtile range of migration timing for sockeye, pink, and chum in
2018, contrasted to the time series averages for the Discovery Islands
and Johnstone Strait, respectively. Based on the comparison of peak
migration dates between the two zones, we estimate that the average
residence time of juvenile salmon in the Discovery Islands for 2018 was
approximately two weeks.

\begin{figure}[H]
\includegraphics[width=0.8\linewidth]{peer_j_migration_dynamics_files/figure-latex/mt-1} \caption{Cumulative catch of juvenile sockeye salmon migrating through the Discovery Islands compared to the average for 2015--2018. Migration curves were predicted by fitting a logistic growth equation to the cumulative percent of sockeye in each year.}\label{fig:mt}
\end{figure}

\begin{longtable}[]{@{}llrrr@{}}
\caption{\label{tab:mtdi} Interquartile range for the cumulative catch of
sockeye, pink, and chum salmon in the Discovery Islands in 2018,
compared to the time-series average . Odd years were excluded from the
TSA calculation for pink salmon due being the ``off'' years in the
outmigration cycle.}\tabularnewline
\toprule
Species & Year & 25\% & 50\% & 75\%\tabularnewline
\midrule
\endfirsthead
\toprule
Species & Year & 25\% & 50\% & 75\%\tabularnewline
\midrule
\endhead
Sockeye & Average & May 26 & May 28 & Jun 04\tabularnewline
~ & 2018 & May 23 & May 23 & Jun 04\tabularnewline
Pink & Average & Jun 05 & Jun 13 & Jun 16\tabularnewline
~ & 2018 & Jun 07 & Jun 12 & Jun 18\tabularnewline
Chum & Average & Jun 06 & Jun 15 & Jun 22\tabularnewline
~ & 2018 & Jun 07 & Jun 12 & Jun 20\tabularnewline
\bottomrule
\end{longtable}

\subsection*{Species Proportions}\label{species-proportions}
\addcontentsline{toc}{subsection}{Species Proportions}

Pink salmon dominated the catch in the Discovery Islands and Johnstone
Strait in 2018, which is the first time observed in the time-series
(Figure \ref{fig:prop}). This may be due to post-smolts being from the
dominant odd-year pink returning broodlines (Krkošek et al. 2011;
Beacham et al. 2012; Irvine et al. 2014) coupled with Fraser River
sockeye from the weak 2016 brood year, which was lowest recorded return
in 100 years (McKinnell et al. 2012; Grant, MacDonald, and Michielsens
2017; Pacific Salmon Commission 2017).

\begin{figure}[H]
\includegraphics[width=0.8\linewidth]{peer_j_migration_dynamics_files/figure-latex/prop-1} \caption{The annual proportion of fish captured in the Discovery Islands and Johnstone Strait combined.}\label{fig:prop}
\end{figure}

\subsection*{Length}\label{length}
\addcontentsline{toc}{subsection}{Length}

Fish lengths varied between regions, species and year (Figure
\ref{fig:length}). Sockeye lengths were 5.07 mm longer than average (Z =
0.62. (Discovery Islands mean = 116 mm, 95\% CI = 113---120; Johnstone
Strait mean = 117 mm, 95\% CI = 113---122). However, pink were shorter
than average in both regions (8.3 \% in the Discovery Islands and 4.4 \%
Johnstone Strait), as well as chum (8.1 \% in the Discovery Islands and
3.6 \% Johnstone Strait). Figure \ref{fig:lengthstatsDI} and Figure
\ref{fig:lengthstatsJS} compare and contrast the average lengths in 2018
against the time-series average.

\begin{figure}[H]
\includegraphics[width=0.8\linewidth]{peer_j_migration_dynamics_files/figure-latex/length-1} \caption{Kernel density distributions of juvenile salmon fork lengths for each year in the selected region. Note that these distributions contain multiple age-classes.}\label{fig:length}
\end{figure}

\textbackslash{}begin\{figure\}{[}H{]}
\includegraphics[width=0.8\linewidth]{peer_j_migration_dynamics_files/figure-latex/lengthstatsDI-1}
\textbackslash{}caption\{The average fork length +/- 95\% C.I. of
juvenile salmon in the Discovery Islands. TSA is the Time Series Average
taken over 2015 -- 2018.\}\label{fig:lengthstatsDI}
\textbackslash{}end\{figure\}

\textbackslash{}begin\{figure\}{[}H{]}
\includegraphics[width=0.8\linewidth]{peer_j_migration_dynamics_files/figure-latex/lengthstatsJS-1}
\textbackslash{}caption\{The average fork length +/- 95\% C.I. of
juvenile salmon in Johnstone Strait. TSA is the Time Series Average
taken over 2015 -- 2018.\}\label{fig:lengthstatsJS}
\textbackslash{}end\{figure\}

\subsection*{Parasite Loads}\label{parasite-loads}
\addcontentsline{toc}{subsection}{Parasite Loads}

Across the Discovery Islands and Johnstone Strait, parasite loads were
11.8 percent less than average (Z = -0.98) The prevalence of motile
(pre-adult and adult life stage) sea lice in 2018 was the lowest
recorded in the time-series (Figure \ref{fig:sealice}). Notably, no
\emph{Lepeophtheirus salmonis} were detected on sockeye in Johnstone
Strait, despite being present in the Discovery Islands. Pink salmon
appeared to have higher counts of \emph{Caligus clemensi} in 2018
compared to chum and sockeye.

\begin{figure}[H]
\includegraphics[width=0.8\linewidth]{peer_j_migration_dynamics_files/figure-latex/sealice-1} \caption{The prevalence (+/-SE) of motile sea lice on juvenile salmon in the Discovery Islands and Johnstone Strait.}\label{fig:sealice}
\end{figure}

\subsection*{Sea Surface Temperature}\label{sea-surface-temperature}
\addcontentsline{toc}{subsection}{Sea Surface Temperature}

Sea-surface temperatures in May and June at QU39 in the northern Strait
of Georgia was 0.39 degrees C warmer than normal (Z = 1.33). Sea surface
temperatures between May and July of 2018 were warmer than the time
series average. (Figure \ref{fig:sst})

\begin{figure}[H]
\includegraphics[width=0.8\linewidth]{peer_j_migration_dynamics_files/figure-latex/sst-1} \caption{Time series of 30 m depth integrated temperature anomalies observed at Hakai Oceanographic Monitoring station QU39. Blue areas represent temperatures that are below normal, red areas represent above normal temperatures at the selected station in 2018. Normal is the solid black line which is a loess regression based on temperatures from 2015-2018. The shaded grey area is 1 SE of the loess regression. The black dots are the daily minimum and maximum temperatures observed over the time series.}\label{fig:sst}
\end{figure}

\section*{References}\label{references}
\addcontentsline{toc}{section}{References}

\hypertarget{refs}{}
\hypertarget{ref-Beacham2012}{}
Beacham, Terry D., Brenda Mcintosh, Cathy MacConnachie, Brian Spilsted,
and Bruce A. White. 2012. ``Population structure of pink salmon
(Oncorhynchus gorbuscha) in British Columbia and Washington, determined
with microsatellites.'' \emph{Fishery Bulletin} 110 (2): 242--56.
doi:\href{https://doi.org/10.2337/diabetes.51.4.1093}{10.2337/diabetes.51.4.1093}.

\hypertarget{ref-Beamish2001}{}
Beamish, R J, and Conrad Mahnken. 2001. ``A critical size and period
hypothesis to explain natural regulation of salmon abundance and the
linkage to climate and climate change.'' \emph{Progress in Oceanography}
49: 423--37.

\hypertarget{ref-Beamish2012}{}
Beamish, R. J., C. Neville, R. Sweeting, and K. Lange. 2012. ``The
synchronous failure of juvenile pacific salmon and herring production in
the strait of georgia in 2007 and the poor return of sockeye salmon to
the Fraser river in 2009.'' \emph{Marine and Coastal Fisheries} 4 (1):
403--14.
doi:\href{https://doi.org/10.1080/19425120.2012.676607}{10.1080/19425120.2012.676607}.

\hypertarget{ref-Grant2017}{}
Grant, Sue C.H., Bronwyn L. MacDonald, and Catherine G.J. Michielsens.
2017. ``Fraser River Sockeye: Abundance and Productivity Trends and
Forecasts.'' North Pacific Anadromous Fish Commission; Fisheries; Oceans
Canada; Pacific Salmon Commission. \url{http://www.npafc.org}.

\hypertarget{ref-Hunt2018}{}
Hunt, Brian P.V., Brett T. Johnson, Sean C. Godwin, Martin Krkošek,
Evgeny A Pakhomov, and Luke A Rogers. 2018. ``The Hakai Institute
Juvenile Salmon Program : Early Life History Drivers of Marine Survival
in Sockeye , Pink and Chum Salmon in British Columbia.'' Institute for
the Oceans; Fisheries; Department of Earth, Ocean; Atmospheric Sciences,
University of British Columbia, Hakai Institute, Earth to Ocean Research
Group, Simon Fraser University, Department of Ecology; Evolutionary
Biology, Univer.

\hypertarget{ref-Irvine2014}{}
Irvine, J.R., C.J.G. Michielsens, M. O'Brien, B.A. White, and M. Folkes.
2014. ``Increasing Dominance of Odd-Year Returning Pink Salmon.''
\emph{Transactions of the American Fisheries Society} 143 (4): 939--56.
doi:\href{https://doi.org/10.1080/00028487.2014.889747}{10.1080/00028487.2014.889747}.

\hypertarget{ref-Krkosek2011}{}
Krkošek, Martin, Ray Hilborn, Randall M. Peterman, and Thomas P. Quinn.
2011. ``Cycles, stochasticity and density dependence in pink salmon
population dynamics.'' \emph{Proceedings of the Royal Society B:
Biological Sciences} 278 (1714). The Royal Society: 2060--8.
doi:\href{https://doi.org/10.1098/rspb.2010.2335}{10.1098/rspb.2010.2335}.

\hypertarget{ref-Margolis1990}{}
Margolis, L., G. W. Esch, A.M. Kuris, and G.A. Schad. 1990. ``The Use of
Ecological Terms in Parasitology (Report of an Ad Hoc Committee of the
American Society of Parasitologists).'' \emph{The Journal of
Parisitology} 68 (1): 131--33.
doi:\href{https://doi.org/10.2307/3281335}{10.2307/3281335}.

\hypertarget{ref-McKinnell2012}{}
McKinnell, Stewart M., Enrique Curchitser, Cornelius Groot, Masahide
Kaeriyama, and Katherine W. Myers. 2012. ``PICES Advisory Report on The
Decline of Fraser River Sockeye Salmon Oncorhynchus nerka (Steller,
1743) in Relation to Marine Ecology.'' PICES Scientific Report.
Vancouver, B.C.: Cohen Commission; North Pacific Marine Science
Organization (PICES). \url{www.cohencommission.ca}.

\hypertarget{ref-PacificSalmonCommission2017}{}
Pacific Salmon Commission. 2017. ``Report of the Fraser River Panel to
the Pacific Salmon Commission on the 2016 Fraser River Sockeye Salmon
Fishing Season.'' Pacific Salmon Commission.
\url{https://www.psc.org/publications/annual-reports/fraser-river-panel/}.

\hypertarget{ref-Peterman2010}{}
Peterman, Randall M, D Marmorek, B Beckman, M Bradford, M Lapointe, N
Mantua, Brian Riddell, et al. 2010. ``Synthesis of evidence from a
workshop on the decline of Fraser River sockeye. June 15-17, 2010. A
Report to the Pacific Salmon Commission.'' August. Vancovuer, British
Columbia: Pacific Salmon Commission.



\end{document}
